\documentclass[10pt]{article}
\usepackage[backend=bibtex]{biblatex}
\usepackage{graphicx}
\usepackage{braket}
\usepackage[pdfencoding=auto, psdextra]{hyperref}
\usepackage{multicol}
\usepackage{geometry}
\usepackage{caption}


 \geometry{
 a4paper,
 total={180mm,257mm},
 left = 15mm,
 top = 20mm, 
 }

\setcounter{secnumdepth}{1}
\usepackage{abstract}
\renewcommand{\abstractname}{}    % clear the title
\renewcommand{\absnamepos}{empty} % originally center
\renewcommand{\thesection}{\Roman{section}} 

\newenvironment{Figure}
  {\par\medskip\noindent\minipage{\linewidth}}
  {\endminipage\par\medskip}




\bibliography{my_bibliography} 
%\bibliographystyle{ieee}

\makeatletter
\renewcommand{\maketitle}{\bgroup\setlength{\parindent}{0pt}
\begin{flushleft}
  \textbf{\@title}

  \@author
\end{flushleft}\egroup
}
\makeatother

\title{\LARGE Biological Sensing with Nitrogen-Vacancy Centres in Diamond}
\date{}
\author{\vspace{3mm}
\large Conor Mc Keever, University College London
}



\begin{document}

\maketitle
\vspace{0.4cm}

\begin{abstract}
\textbf{This is the abstract}
\end{abstract}


\begin{multicols}{2}
\normalsize
\tableofcontents
\section{Introduction}
The nitrogen-vacancy centre in diamond exhibits properties which make it extraordinarily useful for applications in quantum science. The ability to address and red out its spin state optically allows for the exploration of a plethora of possible applications across, chemistry and - as explored in this report - biology. The properties of its host crystal diamond, allow for a high degree of biocompatibility, with diamonds being successfully used in drug delivery and as simple biomarkers. The high standard of fabrication techniques have allowed the systehesis of bulk diamonds and nanodiamonds with suitable high concentrations of NV centres broadening the scope of their potential applications. In this report, the properties of the diamond NV centre are elucidated and their biological sensing applications beyond simple biomarkers are explored. In section II the basic optical and spin properties of the NV$^-$ centre are presented. In section III, the synthesis of (nano)diamonds appropriate for sensing applications is explored and their functionalisation for applications in a biological setting is presented. Finally in section IV, a number of experimental results in the field are presented which encapsulate the breadth of potential sensing applications offered by the NV centre in diamond.

\section{The NV\texorpdfstring{$^-$} Centre in Diamond}
A wide variety of defects occur naturally in diamond. Many of these defects allow the absorption and emission of light \cite{zaitsev2001optical}, often giving their host crystal a natural vivid colour. These colour centres are principally due to site vacancies and impurity defects in the crystal, with elements including boron, silicon, nickel and most commonly nitrogen \cite{wu2016diamond}. The negatively charged nitrogen-vacancy (NV$^-$) centre diamond is of particular interest in sensing technologies not only due to their natural abundance, but also due to a number of favourable properties which they posses. 

\begin{Figure}
  \includegraphics[width=\linewidth]{energy_level.png}
	  \captionof{figure}{\textbf{Energy-level diagram of NV$^-$ colour centre. The ground $\ket{g}$ and excited $\ket{e}$ state spin triplets have a resonant transition wavelength of 638 nm with a radiative lifetime of $\sim$ 25 ns (for nanodiamonds) \cite{doherty2013nitrogen}. The metastable singlet state $\ket{s}$ has a lifetime of $\sim$ 250 ns. The spin triplet sub-levels are Zeeman split by an external magnetic field while a zero-field splitting D has a value of $D=2,870$ MHz \cite{schirhagl2014nitrogen,doherty2013nitrogen}. Figure:\cite{schirhagl2014nitrogen}}}
  \label{fig:energy_level}
\end{Figure}

\subsection{Electronic Structure}
The NV centre localises six electrons at the defect site. The nitrogen atom provides two of its valence electrons while a further three are due to dangling bonds of the diamond's carbon atoms\cite{schirhagl2014nitrogen}. The remaining electron is captured from donor ions in the lattice giving the NV$^-$ colour centre a net negative ($-e$) charge \cite{schirhagl2014nitrogen}. Although the positively charged NV$^+$ and neutral NV$^0$ defects exist, the NV$^-$ is the only variant which is magneto-optically active and is the focus of the vast majority of research \cite{schirhagl2014nitrogen}. Indeed, ionisation to NV$^0$ poses a challenge to the synthesis of shallow NV$^-$ centres \cite{hauf2011chemical}. 




\subsection{Optical and Spin Properties}
The optical properties of NV$^-$ centres are crucial to their application in sensing technologies. The essential features of the system can be described by a simple energy-level diagram of Figure.\ref{fig:energy_level}. The spin triplet ground $\ket{g}$ and excited $\ket{e}$ states lie between the conduction and valence bands and have an energy splitting of $\sim$1.945 eV \cite{schirhagl2014nitrogen}. A metastable spin singlet state $\ket{s}$ lies between $\ket{g}$ and $\ket{e}$ and has a radiative lifetime of $\sim$ 250 ns \cite{schirhagl2014nitrogen}. The spin triplet states are split into three spin sub-levels labelled by $m_S$. The $m_0$ state is lower in energy than the degenerate $m_{\pm1}$ states due to a zero-field splitting D of 2.87 GHz for $\ket{g}$ and 1.42 GHz for $\ket{e}$ triplets \cite{doherty2013nitrogen}. The triplet $m_{\pm1}$ degeneracy is lifted by an external magnetic field and thus the system can be used as a magnetic field probe. The metastable long-lived state $\ket{s}$ is primarily populated from the $\ket{e,m_{\pm1}}$ states while the $\ket{e,m_0}$ state decays more strongly to $\ket{g}$, as a consequence, an optical contrast between $m_0$ and $m_{\pm1}$ of 30\% arises \cite{schirhagl2014nitrogen}. This spin dependent luminescence is the basis of many sensing applications. 

\section{Synthesis and Functionalisation}
\subsection{Bulk Diamond Synthesis}
Synthetic diamonds can be synthesized by a variety of methods. These include high-pressure-high-temperature (HPHT) growth and chemical vapour deposition (CVD). The majority of bulk diamonds are produced by HPHT growth and the method offers a high degree of control over the size and quality of the diamonds \cite{wu2016diamond}. As an artefact of the growth process, most of these diamonds contain nitrogen impurities and have dimensions ranging from micrometers up to a few millimetres. Lattice vacancies can be produced by irradiation with high-energy particles and subsequent annealing allows the formation of NV centres \cite{wu2016diamond}. CVD involves the low pressure growth of diamonds by using carbon-rich gases. The gases are fragmented using a plasma created between two electrodes, the carbon then reassembles into a diamond film \cite{wu2016diamond}. Performing CVD in the presence of a nitrogen gas mixture allows the creation of NV centres in the diamond offering control over the concentration of NV centres \cite{wu2016diamond}. 

\subsection{Nanodiamond Synthesis}
The fabrication of nanodiamonds is possible via a wide variety of methods such as detonation \cite{shenderova2012ultrananocrystalline}and laser ablation \cite{amans2009nanodiamond}, however many are not suitable for applications in sensing. The ideal nanodiamond for sensing applications would contain a highly stable NV centre in a defect free crystalline diamond environment \cite{wu2016diamond}. For this reason, nanodiamonds for quantum sensing are normally prepared by milling of HPTP microdiamonds and CVD. Irradiation and annealing can be used to increase the concentration of NV centres however precise control over the size, shape and NV concentration of nanodiamonds remains a challenge \cite{wu2016diamond}. 

\subsection{Functionalisation}
Once nanodiamonds have been created which fulfil the requirements for a good quantum sensor, it is crucial for biosensing applications that the diamond surface is stable in a biological environment. A number of methods of surface functionalisation exist including the binding of organic proteins and inorganic materials such as silica and polymers \cite{wu2016diamond}. The search for appropriate functionalisation methods is an active area of research and promises make NV centre based sensing possible in a wide range of biological applications.  


\section{Sensing with NV centres}

\subsection{Sensing Protocols}
Sensing involves the measurement of perturbations to a system by interrogation. The measurement of EPR frequency shifts is the principal method of interrogation and a number of protocols to this end exist. 

\subsubsection{Continuous Measurement}
Continuous measurement involves the measurement of the EPR spectrum across a suitable range and the subsequent fitting of any resonances observed. The application of an external magnetic field will for instance cause the resonance to shift its centre position. Sensitivity can be increased by observing variations in fluorescence intensity and by collecting photons from many NV$^-$ centres, this however is detrimental to the nanoscale sensing characteristics of a single NV$^-$ centre \cite{schirhagl2014nitrogen}. The sensitivity of this measurement technique suffers since the system is continuously being measured, nevertheless sensitivities below kHz/$\sqrt{\textrm{Hz}}$ have been achieved \cite{acosta2010broadband}. A significant body of work has extended the possibilities of sensing using this technique (see \textit{e.g.}\cite{haberle2013high,schoenfeld2011real}) including the sensing of vector fields using multiple NV$^-$ centres with different orientations \cite{maertz2010vector}.

\subsubsection{Pulsed Measurement}
In pulsed measurements a pump probe scheme is employed in which the system evolves in the perturbing field without any interrogation. The lack of optical pumping during the measurement phase increases sensitivity over long coherent evolution times $\tau$ \cite{maze2008nanoscale}. Furthermore, a wide variety pulse sequences have been developed in recent years, many of which can be employed to make extremely sensitive measurements, even in the presence environmental fluctuations \cite{schirhagl2014nitrogen}.

\subsubsection{Relaxometry}
Magnetic resonance relaxometry relies upon the fact that the spin relaxation times T$_1$ and T$_2$ are dependant on the environmental degrees of freedom of the perturbing system. One particular implementation of relaxometry is the measurement of fluctuating noise in the environment and as in the case of pulsed measurement, many protocols and techniques exist for the sensitive measurement and characterisation of spin relaxation times \cite{steinert2013magnetic}.


\subsection{Applications in Biological Sensing}
NV centres have properties which make them highly attractive for applications in biology. Nanodiamonds are considered stable and biocompatible, having less cytotoxicity than many other materials \cite{vaijayanthimala2012long,schrand2007differential}, which makes them ideal candidates for in vivo and in vitro applications. Nanodiamonds have been used in drug delivery in living systems \cite{huang2007active} which is a testament to their biocompatibility, however questions remain over the nanotoxicity (toxicity due to nanometre sizes) of nanodiamonds \cite{schrand2007differential}. Their inert nature allows NV centres to probe biological systems non-invasively and their room temperature operation offers advantages over sensing systems which require external environments such as cryogenic temperatures \cite{wu2016diamond}. 

\subsubsection{Electric Field Sensing}
The measurement of AC electric fields at sensitivities of $\sim$140 V/cm/$\sqrt{\textrm{Hz}}$ has been demonstrated \cite{dolde2011electric}. This is possible due to a Stark shift in the spin sub-levels of the NV centre producing optically observable shifts in resonance peaks in the presence of an external electric field \cite{dolde2011electric}. Electric field measurements implemented using nano-diamond sensors may allow for probing of cell membrane potentials which experience a strong potential drop (40-80 mV) across the cell boundary \cite{schirhagl2014nitrogen}.

\subsubsection{Optical Trapping}
It has been shown that it is possible to optically trap nanodiamonds and simultaneously perform sensing protocols \cite{horowitz2012electron,geiselmann2013three}. Three dimensional spatial manipulation of individual nanodiamonds was demonstrated using optical tweezers. Remarkably, it was also shown that by adjusting the optical tweezer's polarisation, the NV centre axis direction could also be controlled \cite{geiselmann2013three}. This high degree of control could allow the precise imaging of structures in living cells such as nuclear spins of proteins \cite{geiselmann2013three}. Similar optical trapping of an ensemble of NV centres has also been demonstrated \cite{horowitz2012electron}. By measuring ESR spectra, it was shown that dc magnetometry was possible and measurements achieved a sensitivity of 50 $\mu$T/$\sqrt{\textrm{Hz}}$ \cite{horowitz2012electron}. 

\subsubsection{Wide-Field Microscopy}
Arrays of shallow NV$^-$ centres in diamond films have been used in wide-field microscopy experiments in which the fluorescence of an entire array centres could be monitored simultaneously \cite{le2013optical}. Using this method, vector field maps created by magnetic particles produced in immobilised bacteria on the surface of the diamond were rapidly reconstructed with subcellular resolution \cite{le2013optical}. In one experiment \cite{le2013optical}, magnetotactic bacteria (bacteria with magnetic moments) were imaged using an wide-field array of NV centres embedded 10 nm below the surface of a pure diamond chip. Four independent measurements enabled determination of the vector components of the bacterial magnetic moments as shown in Figure \ref{fig:wide_field}. Although other methods exist for imaging magnetotactic bacteria, only NV centre arrays operate at ambient temperatures, opening the possibility of studying real time dynamics of these cells \cite{le2013optical}. A similar wide-field microscopy approach has also been proposed to image neuron activity \cite{hall2012high}. 

\begin{Figure}
  \includegraphics[width=\linewidth]{wide_field.jpg}
	  \captionof{figure}{\textbf{a. Bright field image of magnetotactic bacterium. b.-d. (f.-h.) Wide-field microscopy images (fits) of a single magnetotactic bacterium, independent magnetic field measurements allow for the reconstruction of the bacterial magnetic moment. e. SEM image.  Figure:\cite{le2013optical}}}
  \label{fig:wide_field}
\end{Figure}

\subsubsection{Scanning Magnetometry}
The basic idea of scanning magnetometry with NV centres is to embed an diamond at the tip of an AFM scanning probe cantilever and to scan across a sample. This concept was realised experimentally \cite{balasubramanian2008nanoscale}. Figure. \ref{fig:magnetometry} a shows the setup of this apparatus. In the experiment, a nanometre scale magnetic structure was imaged (see Figure. \ref{fig:magnetometry} c). Microwaves were tuned into resonance with the NV centre for a particular magnetic field strength and projection. As the tip was scanned across the structure, the resonance changes and the fluorescence disappeared causing the magnetic structure to appear as a shadow \cite{balasubramanian2008nanoscale}. Magnetic field resolution of 0.5 mT was obtained and was limited by the oscillatory motion of the cantilever tip. This technique could be utilised in biological imaging applications, particularly where electron or even nuclear spin imaging is desired \cite{balasubramanian2008nanoscale}. 

\begin{Figure}
  \includegraphics[width=0.9\linewidth]{magnetometry.jpg}
	  \captionof{figure}{\textbf{a. Illustration of NV centre diamond on the tip of scanning probe cantilever, the method could be used to image neighbouring spins. b. Microwaves tuned on resonance with the NV centre produces fluorescence, the AFM tip outline is visible. c. AFM and optical fluorescence images of a nanoscale magnetic structure, the structure causes the NV centre to go off resonance and the image goes dark. A dark line of resonance is observed at 5 mT. Figure:\cite{balasubramanian2008nanoscale}}}
  \label{fig:magnetometry}
\end{Figure}



\subsubsection{Thermometry} 
It has been shown \cite{acosta2010temperature} that the zero {}field splitting parameter (ZFS), D has a significant dependence on temperature T. Across a variety of samples it was found that $dD/dT = -74.2(7)$ kHz/K while the transverse ZFS parameter E had a dependence $dE/dT = -1.4(3)\times10^{-4}$ $K^{-1}$ \cite{acosta2010temperature}. This effect is due thermal expansion induced strain of the local crystalline environment. CLearly this effect limits the sensitivity of NV$^-$ centres in noisy thermal environments however it can also be exploited as a method to use the NV$^-$ centre as a nanoscale thermometer \cite{toyli2013fluorescence,neumann2013high,kucsko2013nanometre}. Using a high purity bulk sample of diamond it was shown \cite{kucsko2013nanometre} that temperature variations of 1.8 mK at a sensitivity of 9 mK Hz$^{-1/2}$ are measurable using a pulse sequence protocol. Furthermore the local thermal environment could be resolved at 200 nm length scales. This technique has allowed the mapping of temperature gradients inside cells with the use of an ensemble of ($\sim$500) NV centres) \cite{kucsko2013nanometre}. In these experiments the ZFS splitting of a number of spatially distributed NV centres was monitored continuously, the technique was demonstrated in living cells which were shown to remain viable after measurement \cite{kucsko2013nanometre}. It was also shown that the introduction of a gold nanoparticle to the cell which could be heated using a laser, allowed for the simultaneous control and measurement of cell temperature, paving the way for a wide variety of applications in biology \cite{kucsko2013nanometre}.

\begin{Figure}
  \includegraphics[width=\linewidth]{thermometry.png}
	  \captionof{figure}{\textbf{Illustration of simultaneous temperature control and measurement inside a cell. The cell environment is heated using a green laser incident on a gold nanoparticle while NV centres are used to measure the local cellular temperature. Figure:\cite{kucsko2013nanometre}}}
  \label{fig:energy_level}
\end{Figure}


\subsubsection{Strain and Pressure Sensing}
The effects of hydrostatic pressure on the behaviour of NV$^-$ centres has been investigated \cite{doherty2014electronic}. It was found that the variation of the ZFS parameter D with pressure P was highly linear at $dD/dP = 14.58(6)$ MHz/GPa \cite{doherty2014electronic}. The measurements were performed in a diamond anvil cell capable of reaching extreme pressures and the technique may offer improved sensitivity over current pressure sensing techniques in these environments \cite{doherty2014electronic} however are unlikely to be sufficiently sensitive for applications in biology. 

\subsubsection{Nuclear Magnetic Resonance}
The mapping of nuclear spins is an important problem in many areas of physics. Understanding the decoherence mechanisms of spin systems would benefit greatly from a single magnetic spin probe in the form of a NV centre. In biology, the structure determination of proteins and other complex molecules would be profoundly impacted by a NV nuclear spin sensor \cite{schirhagl2014nitrogen}. Experiments have shown that it is possible for NV centres to sense the nanotesla field fluctuations from protons located in a nearby organic sample allowing NMR measurements on the nanometre scale \cite{mamin2013nanoscale}. In these experiments, a highly sensitive pulse sequence was employed to measure proton fluctuations of a layer of polymethyl methacrylate (PMMA) and a clear signature of proton fluctuations was observed \cite{mamin2013nanoscale}. Having demonstrated a proof of principle, the main technical challenges which remain involve the reduction of NV centre depth below the diamond surface and the efficient measurement of fluorescence \cite{mamin2013nanoscale}. 


\subsection{Challenges and Outlook}
One of the greatest challenges in developing more sensitive NV centres for sensing applications is the synthesis of shallow (1-10 nm)NV centres near the diamond surface \cite{schirhagl2014nitrogen}. The outstanding problem with shallow NV centres is the loss of an electron due to edge effects and the conversion to the neutral NV$^0$ centre, which is not optically active \cite{schirhagl2014nitrogen}. Various methods have been proposed to circumvent this problem, including surface termination by oxygen or fluorine \cite{cui2013increased,fu2010conversion}, however no NV centre shallower than 2 nm has been observed \cite{schirhagl2014nitrogen}. Furthermore, shallow NV centres suffer from reduced spin relaxation times, particularly in nanodiamonds, an effect which is not yet fully understood \cite{schirhagl2014nitrogen}. The mitigation of these surface effects will be crucial to increasing NV centre sensitivity to the level needed for many proposed applications. For in-vivo applications of NV centre sensing, a full characterisation of diamond and nanodiamond toxicity will be required, this may involve optimisation of nanodiamond shape and size or improved functionalisation of surfaces \cite{wu2016diamond}. More efficient fluorescence imaging and spectroscopy methods will also contribute to their development. 

\section{Conclusion}{}
{}

\printbibliography

\end{multicols}

\end{document}

