\documentclass[12pt]{article}
\usepackage[backend=bibtex,style=verbose-trad2]{biblatex}

\bibliography{my_bibliography} 
%\bibliographystyle{ieeetr}

\begin{document}
\section{Introduction}
\section{The NV$^-$ Centre in Diamond}
Aim: Explain basic physics well so that it is easy to refer back to in the remainder of the report. 

1.explain the basic physics of NV centres - how they are manipulated addressed read out, energy level diagrams, etc.

2. Explain how they can be used as quantum sensors for varius types of parameters, magneitc field, electric fields, strain temperature etc.

3. Explain the possibilites in fabrication of sensing devices and how amenable they are to sensing in the biological setting. 

\subsection{Basic physics}
\subsection{Applications in Quantum Sensing}
\subsection{Fabrication}


\section{Biological Sensing with NV$^-$ centres}
Aim: Focus on the ways in which the sensing possibilites described in the previous section can be used in biological settings. The advantages/disadvantages/challenges. Give an overview of some of the seminal work done in the area and focus in on the recent developments in the field. many examples. 

\section{Conclusion}
\cite{maletinsky2012robust}

\newpage
\printbibliography

\end{document}

